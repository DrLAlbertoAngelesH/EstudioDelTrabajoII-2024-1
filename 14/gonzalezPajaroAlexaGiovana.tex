\section{Alexa Giovana Gonzalez Pajaro}
\subsection{Definiciones}
\begin{enumerate}
    \item Exacto: Es la cercanía que tiene una medición respecto a su objetivo.
    \item Preciso: Es la variación que se puede presentar en un fenónemo físico.
    \item Movimiento: Estado de los cuerpos mientras cambian de lugar o de posición.
    \cite{RAE}
    \item Estudio: Esfuerzo que pone el entendimiento aplicándose a conocer algo.
    \cite{RAE}
    \item Trabajo: Es una actividad esencial que implica esfuerzo físico o mental con el objetivo de producir bienes o servicios para la sociedad.
    \cite{RAE}
    \item Estudio de movimientos y tiempos:Es el análisis de métodos, materiales, herramientas e instalación utilizada o que se ha de utilizar en la ejecución de un trabajo.
    \cite{RAE}
    \item Analisis: Distinción y separación de las partes de algo para conocer su composición.
    \cite{RAE}
    \item Sistema: Es la composición de varios elementos que están conectados entre sí para lograr un objetivo específico.
    \cite{RAE}
    \item Predeterminar: Determinar o resolver con anticipación algo.
    \cite{RAE}
    \item Medición de tiempos de metodos (MTM): MTM es la abreviatura de Methods-Time Measurement. Métodos-Medición del Tiempo, significa que el tiempo requerido para realizar una tarea específica depende del método elegido para realizarla.
    \cite{RAE}
    \item Sistema de tiempos predeterminados: se refiere a un cuerpo organizado de información, procedimientos, técnicas y tiempos de movimiento empleado en el estudio y evaluación de los elementos del trabajo manual. 
    \cite{RAE}
    \item Estudio de movimientos: es el analisis cuidadoso de los movimientos corporales que se emplean para realizar una tarea. Su propósito es eliminar o reducir movimientos ineficientes para facilitar y acelerar los movimientos eficientes.
    \cite{RAE}
    \item Estudio de tiempos: es una técnica de medición del trabajo empleada para registrar los tiempos y ritmos de trabajo correspondientes a los elementos de una tarea definida, efectuada en condiciones determinadas y para analizar los datos a fin de averiguar el tiempo requerido para efectuar la tarea según una norma de ejecución preestablecida.
    \cite{RAE}
    \item Naturaleza del trabajo: El trabajo es una actividad vital del hombre en la que transforma la naturaleza en productos, y se convierte en la base de apropiación y resultado del trabajo del hombre.
    \cite{RAE}
    \item Medición: Es el proceso a través del cual se compara la medida de un objeto o elemento con la medida de otro.
    \cite{RAE}
    \item Método: Es el conjunto de procedimientos, reglas, operaciones o técnicas que se siguen de forma ordenada y sistemática para alcanzar un objetivo
    \cite{book}
    \item Therblings: Son movimientos estándar que se utilizan para analizar y mejorar los procesos productivos. A través de su clasificación y análisis, se pueden identificar los movimientos que son innecesarios o que pueden ser optimizados para aumentar la eficiencia en el proceso productivo.
    \cite{book}
    \item Alcanzar: Es el movimiento manual básico efectuado con el fin predominante de transportar la mano o los dedos a un destino.
    \cite{RAE}
    \item Mover:Es el movimiento manual básico efectuado con el fin predominante de transportar un objeto a un destino con dedos o mano.
    \cite{RAE}
    \item Muestreo: Selección de una pequeña parte estadísticamente determinada, utilizada para inferir el valor de una o varias características del conjunto.
    \cite{RAE}
    \item Representativo/va: Que sirve para representar algo.
    \cite{RAE}
    \item Inferir: Deducir algo o sacarlo como conclusión de otra cosa.
    \cite{RAE}
    \item Analizar: acción de comprender un determinado fenómeno hasta el momento desconocido.
    \cite{RAE}
    \item Muestreo de trabajo: es una técnica que se utiliza para investigar las proporciones del tiempo total dedicada a las diversas actividades que componen una tarea, actividades o trabajo.
    \cite{book}
    \item Muestreo discreto:se aplica cuando existe un número restringido de personas con las mismas características de la población objetivo y no existe mucho tiempo para llevar a cabo la investigación.
    \cite{book}
    \item Muestreo continuo: Un espacio muestral continuo se basa en los mismos principios, pero tiene un número infinito de elementos en el espacio. En otras palabras, no puede escribir el espacio de la misma manera que escribiría el espacio de muestra para una tirada de dado.
    \cite{book}
    \item Muestra representativa: Es el conjunto de observaciones realizadas sobre una población. Además, debe de tener el tamaño suficiente y las observaciones tienen que haber sido realizadas sobre todos aquellos elementos de la población que poseen ciertas particularidades.
    \cite{RAE}
    \item Estudio de tiempos convencional: Es una técnica de medición del trabajo empleada para registrar los tiempos y ritmos de trabajo correspondientes a los elementos de una tarea definida, efectuada en condiciones determinadas y para analizar los datos a fin de averiguar el tiempo requerido para efectuar la tarea según una norma de ejecución preestablecida.
    \cite{book}
    \item Estudio de tiempos no convencional: Este estudio consiste en conocer y analizar de forma crítica las metodologías y tiempos de nuestros procesos de producción, es decir, por qué se hace así cada proceso y el tiempo que se tarda en realizarlo actualmente.
    \cite{book}
    \item Muestra continua: Se basa en los mismos principios, pero tiene un número infinito de elementos en el espacio. En otras palabras, no puede escribir el espacio de la misma manera que escribiría el espacio de muestra.
    \cite{book}
    \item Proporción: Disposición, conformidad o correspondencia debida de las partes de una cosa con el todo o entre cosas relacionadas entre sí.
    \cite{book}
    \item  Demora: Tardanza, dilación.
    \cite{RAE}
    \item Determinar: Decidir algo, despejar la incertidumbre sobre ello.
    \cite{RAE}
    \end{enumerate}
\bibliographystyle{apalike}
\bibliography{14/referencias}