\section{}
\subsection{Conceptos:}
\begin{enumerate}
\item \textbf{Ingeniería de Métodos:} Incrementa la productividad y la confiabilidad, asícomo reduce los costos unitarios de producción, a través del estudio del trabajo y la medición de este.
\item  \textbf{Therbligs:} Simbología creada por los esposos Gilbreth, movimientos fundamentales.
\item \textbf{Diagrama Gantt:} Creada por Henry Laurence Gantt, para programar el equipo de producción.

\item \textbf{Línea de espera:} : Determina el numero optimo de estaciones de servicio.

\item \textbf{Estudio de tiempos:} Estudio de los movimientos del cuerpo humano que se utilizan para realizar una labor determinada, con el objetivo de hacerlos mas eficientes.
\item \textbf{Tiempo estándar:} : El tiempo estándar se define como el tiempo que necesita un operador cualificado preparado y entrenado para ejecutar una operación, trabajando a una velocidad normal.

\item \textbf{Tiempo normal:} Es la multiplicación del tiempo promedio por el factor de calificación o valoración. Promedio de los datos cronometrados.

\item \textbf{Personal calificado:} persona competente y capacitada que cumple los requisitos y entrenamiento en un campo específico aceptable para la autoridad competente.

\item \textbf{Holguras:} Es la diferencia entre su tiempo de determinación más lejana y su tiempo de terminación más cercana de una actividad.

\item \textbf{Método regreso a cero:} : El reloj se acciona al comienzo del primer elemento del primer ciclo, al final de cada elemento el reloj muestra el tiempo para cada elemento y se regresa a cero.

\item \textbf{Método continuo:} Se caracteriza por la aplicación de una carga ininterrumpida, sin pausa o períodos de descanso durante el trabajo. La duración del trabajo suele ser prolongada

\item \textbf{Muestreo:}
    \\Acción de escoger muestras que describan de manera exacta las características de un conjunto de datos que permitirán deducir y sacar conclusiones del fenómeno a estudiar.

\item \textbf{Muestro del trabajo:} Herramienta para disminuir el costo que se presenta en el estudio continuo del tiempo.

\item \textbf{Muestreo discreto:} Implica seleccionar elementos específicos de una población finita o contable, donde cada elemento tiene una probabilidad asociada de ser elegido.

\item \textbf{Muestreo continuo: }
    \\Selección de elementos de una población infinita, donde los valores pueden tomar cualquier valor dentro de un rango específico, como la altura o el tiempo.

\item \textbf{Estudio de tiempos convencional: }
    \\Es una muestra continua de n ciclos (Suponiendo que la distribución estadística es normal).


\item \textbf{Distribución estadística normal: }
    \\Forma de distribución estadística simétrica con forma de campana, donde la mayoría de los datos se concentran cerca de la media.

\item \textbf{Estudio de tiempos no convencional: }
    \\Es una muestra discreta (Suponiendo que la distribución estadística es binomial).


\item \textbf{Distribución estadística binomial: }
    \\Modelo estadístico que describe la probabilidad de obtener un número específico de éxitos en un número fijo de ensayos independientes, donde cada ensayo tiene dos resultados posibles: éxito o fracaso.

\item \textbf{Pronostico: }
    \\Valor que se cree obtener.

\item \textbf{Estimación: }
    \\Analisis a traves de operaciones.

\end{enumerate}