\lhead{\begin{tikzpicture}[remember picture, overlay]
    \node [anchor=100,inner sep=0] (imagenIZQUIERDA) at (current page header area.north){\includegraphics[width=18cm]{img/Encabezado.PNG}};
    \end{tikzpicture}}
    \rhead{Ángeles-Hurtado}
    \rfoot{\begin{tikzpicture}[remember picture, overlay]
    \node [anchor=140,inner sep=0] (imagenDERECHA) at (current page footer area.south){\includegraphics[width=18cm]{img/Foot.PNG}};
    \end{tikzpicture}}
    %----------------------------------------------------------------------------------------
    \lfoot{ \thepage}
    % \renewcommand{\labelenumi}{\alph{enumi}.)} 
    %----------------------------------------------------------------------------------------
    %----------------------------------------------------------------------------------------
    %	TITLE SECTION
    %----------------------------------------------------------------------------------------
    
    \setlength{\droptitle}{-5\baselineskip} % Move the title up
    \title{\textbf{Estudio de tiempos y movimientos en el ensamble de un circuito electrónico utilizando diferentes métodos para su optimización }} % Article title
    
     \author{ 
     \textsc{Sáenz-Saavedra, Kenia Paola}\\ 
    %  Afiliación:
     \texttt{ Instituto Tecnológico de Querétaro } \\ 
     \texttt{Tecnológico Nacional de México} \\ 
     \texttt{Querétaro, México}\\ 
     \texttt{l22140911@queretaro.tecnm.mx} 
     \and 
     \textsc{Ángeles-Hurtado, Luis Alberto}\\ 
    %  Afiliación:
     \texttt{ Instituto Tecnológico de Querétaro } \\ 
     \texttt{ Tecnológico Nacional de México } \\ 
     \texttt{Querétaro, México}\\ 
     \texttt{alb3rt0.ah@gmail.com} 
    }
    
    
    %----------------------------------------------------------------------------------------
    
    % \begin{document}
    
    % Print the title
    \maketitle
    \thispagestyle{fancy}
    
    %----------------------------------------------------------------------------------------
    %	ARTICLE CONTENTS
    %----------------------------------------------------------------------------------------
    
    % \section*{Resumen}
    % \textit{Palabras clave:}
    % El resumen (ancho de página) deberá contener entre 100 y 200 palabras tipo Adobe Devangari 11 puntos.
    
    \begin{abstract}
    \noindent         
    \end{abstract}
    % 
    % 
    \textbf{\textit{Palabras clave}}: {Optimización, tiempo estándar.}
    % \keywords{First keyword should be the corresponding to the research area according with the authors guide. Maximum of 6 keywords.}
    \begin{itemize}
        \item Estándar, preciso, exacto, optimización, proceso.
    \end{itemize}
    
    \section{Introducción}
    
    %begin{itemize}
    
    El estudio de tiempos y movimientos es una herramienta la cual sirve para determinar los tiempos estándar de cada una de las operaciones que componen cualquier proceso, así como para analizar los movimientos que son realizados por parte de un operario para llevar a cabo dicha operación.
    
    A finales del siglo XIX, Frederick Taylor comenzó a estudiar los tiempos asociados con actividades laborales y desarrolló el concepto de tarea. Motivados por los estudios de tiempos de Taylor, alrededor del mismo periodo, la pareja de esposos Frank y Lillian Gilbreth condujeron estudios de movimientos (Krenn, 2011) que complementaron el trabajo de Taylor sobre estudios de tiempos.\cite{andrade2019estudio}
    
    El ensamble que se analizará en este estudio de tiempos es un circuito electrónico, este se puede definir como una colección de elementos eléctricos interconectados de alguna forma específica. Este circuito se estudiará a través de un Método de tiempos predeterminados por el grupo de 4to semestre de la carrera de ingeniería industrial. Consiste en una base de datos de tiempos de movimientos básicos.Se asignan tiempos estándar a los elementos básicos del trabajo. Se asignan a los movimientos fundamentales y a grupos de movimientos que no se pueden evaluar con precisión mediante los procedimientos ordinarios de estudio de tiempos con cronómetro. También son el resultado de estudiar una muestra grande de operaciones diversificadas con un dispositivo de ritmo como una cámara de filmación o vídeo-grabación, capaz de medir elementos muy cortos. \cite{parra2020analisis}
    
    El estudio que se realiza en este proyecto integrador tuvo cómo objetivo principal llevar a la práctica los conocimientos adquiridos a lo largo de estos 4 semestres, involucrando las materias y talleres ya cursados en la carrera. De igual manera a través de este ensamble desarrollar nuevas habilidades adquiridas en la materia de Estudió del Trabajo II. Aplicando lo anterior mencionado, obtener buenos resultados a través del estudio, buscando  una optimización en este proceso, que es buscar la mejor manera de realizar una actividad, lo más eficientemente posible, con la menor cantidad de recursos.
    
    El estudio de tiempos con cronómetro de forma tradicional, representa la técnica más utilizada como elemento de medición de las tareas, encontrándose más del 89\% de los trabajos desarrollados bajo ésta técnica.
    La aplicación del estudio de tiempos y movimientos sigue teniendo vigencia en la actualidad, como lo demuestran las 66 investigaciones realizadas entre los años 2010–2016, las cuales aplicaron las técnicas de medición del trabajo en sus formas tradicionales de muestreo del trabajo, estudio de tiempos con cronómetro y estándares de tiempo predeterminados.
    
    %endi{itemize}
     
    % Define estudio de tiempos y movimientos
    % define que es ensamble
    % define que es circuito electronico
    % define el metodo de tiempos predeterminados
    % define optimización
    % 
    % 
    \section{Justificación}
    
    %\begin{itemize}
    El entorno global ha llevado a las organizaciones a buscar la mejora de sus procesos por medio de la identificación y eliminación en forma gradual de las actividades que no generan valor a sus productos y procesos. Estas actividades representan costos operacionales que se traducen en despilfarros de tiempo, materiales, espacio y demás recursos organizacionales. Una de las técnicas más utilizadas para superar dichas deficiencias y elevar la productividad de los trabajadores es el estudio del trabajo. \cite{castiblanco2016que}
    Existen una diversidad de criterios para definir y de este modo clasificar a las empresas como micro, pequeñas, medianas y grandes, estos criterios son diferentes, dependiendo del país o entidad que las define y clasifica, aunque no se puede proporcionar un número exacto, se puede afirmar con seguridad que existen varios millones de empresas de manufactura a nivel global. La industria manufacturera es una parte crucial de la economía mundial, impulsando el crecimiento y el desarrollo económico en muchas regiones del mundo.\cite{saavedra2008caracterizacion}
    En diciembre de 2023, México contaba con 611.331 establecimientos relacionados con el sector manufacturero. El Estado de México era la entidad federativa con la mayor cantidad de locales de este tipo, albergando cerca del 11\% del total. El estado de Querétaro en México es un importante centro industrial y manufacturero, particularmente en sectores como la industria automotriz, aeroespacial, electrónica y de maquinaria. Aproximadamente, en Querétaro existen 7,0000 empresas de manufactura, considerando las unidades económicas registradas en censos y la información de diferentes organismos y clusters industriales. 
        
    
    %\end{itemize}
    % 
    % 
    \section{Descripción del problema}

   % \begin{itemize}
   %     \item Se debe describir la desviación o diferencia del %``es'' con respecto al ``debe ser''.
   %    \item Se debe hacer alusión a la incógnita científica*.
   %     \item Debe de tener Referencias científicas, URL, tesis, etc.
    %\end{itemize}
    
    %\textbf{*La incógnita científica es el elemento cuya solución incrementa el conocimiento científico.}
    Como ya se mencionó anteriormente, el objetivo principal del estudio de tiempos y movimientos es aumentar la eficiencia y reducir el desperdicio de tiempo y recursos en un proceso . Al analizar los procesos educativos, se pueden identificar las actividades que no agregan valor al aprendizaje y eliminarlas o simplificarlas.
    Los estudiantes, al tener la capacidad de realizar este proyecto desarrollaran capacidades y competencias que los posicionaran en un nivel como egresados.
    % 
    % 
    \section{Fundamentación teórica}
    
    %Es la parte medular y de mayor discusión, deberá ser la %fundamentación de la hipótesis, por tanto se deberá señalar %claramente la razón de la suposición y fundamentación de la %misma. Únicamente referencias científicas.
    %\begin{itemize}
     %   \item Se debe de retomar el tema que se planteo en la introducción, pero ahora profundizando para clarificar la incógnita científica y se pueda plantear la hipótesis.
      %  \item Se debe de retomar la descripción del problema, pero ahora a profundidad del (los) objeto(s) de estudio. 
       % \item Se debe de profundizar en las metodologías que se ha usado para el estudio del tema.
       % \item Referencias solo de artículos y libros científicos.
    %\end{itemize}
    El estudio de tiempos y movimientos es una herramienta crucial en la ingeniería industrial, especialmente en el ámbito de la manufactura de productos electrónicos, como los circuitos electrónicos. Este método se utiliza para analizar y optimizar los procesos de producción, mejorando la eficiencia y reduciendo los costos operativos. Involucra la medición detallada del tiempo que toma completar cada tarea en el proceso de ensamble. Este análisis permite identificar y eliminar ineficiencias, estableciendo estándares de tiempo para cada operación. Se enfoca en la simplificación y optimización de los movimientos necesarios para realizar una tarea. Se busca minimizar el esfuerzo y el tiempo empleado en movimientos innecesarios, mejorando la ergonomía y la productividad del trabajador.
    
    Algunos métodos de optimización son el (MTM) Que es el método de medición del tiempo. El MTM es una técnica que descompone cada operación en movimientos básicos y asigna un tiempo estándar a cada uno. Este método es particularmente útil para el análisis de procesos complejos como el ensamble de circuitos electrónicos. 
    También nos podemos encontrar con otro método como es la  Simulación y Modelado: La simulación mediante software como Simio permite modelar diferentes escenarios de producción y evaluar sus impactos en la eficiencia. Al simular el proceso de ensamble, es posible identificar cuellos de botella y probar diversas configuraciones de línea antes de implementarlas físicamente. \cite{breznik2023assembly}.
    
    De igual manera tenemos los Algoritmos Genéticos y Just-in-Time (JIT): La combinación de algoritmos genéticos con estrategias JIT optimiza la programación de la línea de producción y la alimentación de partes. Esto reduce tiempos muertos y exceso de inventario, mejorando la eficiencia global del proceso de ensamble. \cite{zhang2021research}.
    
    En la industria de ensamblaje de circuitos electrónicos, los estudios de tiempos y movimientos se aplican para:
    
   \begin{itemize}
       \item Balance de Línea de Ensamble: Distribuir equitativamente las tareas entre las estaciones de trabajo para minimizar los tiempos de ciclo y los periodos de inactividad. Esto se logra mediante el uso de herramientas como MTM y simulación.
       \item Optimización de Movimientos: Análisis detallado de los movimientos realizados por los operarios para ensamblar componentes, eliminando movimientos innecesarios y mejorando la disposición de las estaciones de trabajo para facilitar el acceso a herramientas y materiales. 
       \item Implementación de Sistemas JIT: Asegurar que los componentes y materiales necesarios lleguen justo a tiempo para ser ensamblados, reduciendo la acumulación de inventarios y los tiempos de espera. Esto se puede optimizar mediante simulaciones y algoritmos genéticos que ajusten dinámicamente las programaciones de producción.
       
    La metodologia de las 5´S es un sistema d eegstión de calidad y productividad originado en Japón cuyo objetivo es mejorar el ambiente de trabajo y la eficiencia organizacional. Esta metodología sigue un proceso establecido en cinco pasos, cuyo desarrollo implica la asignación de recursos, la adaptación a la cultura de la empresa y la consideración de aspectos humanos:
    
    \begin{itemize}
        \item Seiri (Clasificación): Consiste en separar lo neceario de lo innecesario y eliminar lo que no se necesita en el lugar de trabajo.
        \item Seiton (Orden): Implica organizar los elementos esenciales de manera sistematica y eficiente para facilitar su acceso y su uso. cada cosa debe tener un lugar especifico, lo que aumenta la productividad al reducir los tiempos de busqueda y movimineto, y mejora la seguridad al minimizar los obstaculos.
        \item Seiso ( Limpieza): Mantener un ambiente de trabajo limpio y ordenado para prevenir accidentes y mejorar la eficiencia. La limpieza debe de ser una responsabilidad compartida entre todos los empleados para asegurar un entorno laborar seguro y eficiente.
        
   \end{itemize}




    % 
    % 
    \section{Hipótesis}
    
   % Es la suposición con fundamento científico relativa a la solución del problema, necesidad o de cómo se aprovecha la oportunidad con la incógnita científica y se fundamenta con: 1. Una suposición (en afirmativo o negativo) y ésta deberá vincularse con:
    %2. La fundamentación científica que deberá ser precisa 3. Una entidad de comparación para probar la suposición y
    %4. La variable con que se califica o cuantifica la comparación o se prueba la hipótesis.
    Los alumnos elaborarán un proyecto integrador en el cual demostrarán los resultados obtenidos después de haber realizado un  análisis sobre el estudio de tiempos y movimientos, obteniendo el tiempo estándar. A partir de este proyecto adquirirán nuevas habilidades y competencias, así como también pondrán en practica las ya adquiridas a lo largo de los 4 semestres de clases.
    %\begin{itemize}
     %   \item Se debe de identificar claramente la suposición científica
      %  \item Se debe de identificar claramente el fundamento científico
       % \item Se debe identificar claramente la variable de respuesta
        %\item Se debe identifican claramente las realidades o modelos contrastantes
        %\item Se debe de establecer las variables asociadas, explicativas o que tienen relación funcional con la variable de respuesta
    %\end{itemize}
    % 
    % 
    \section{Objetivo}
    
    Realizar un estudio de tiempos y movimientos de un ensamble eléctrico para a partir de este análisis eliminar los movimientos ineficientes, eliminar tiempos muertos y aumentar la productividad, de esta manera comprobando la hipótesis planteada.
    
    
    \subsection{Objetivos específicos }
    
    \begin{itemize}
        \item Diseñar, mejorar e integrar sistemas productivos de bienes y servicios aplicando tecnologías para su optimización
        \item Diseñar, implementar y mejora sistemas de trabajo para elevar la productividad.
        \item Calcular el tiempo estándar que requiere un operador capacitado para el ensamblaje de un circuito electrónico.
        \item Desarrollar una guía de emergencia en la cual se establezcan los posibles riesgos de la locación donde se desarrolló el ensamblaje.
        \item Buscar la manera más económica de realizar la tarea.
        \item Realizar un estudio de tiempos y movimientos analizando todos los factores posibles.
    \end{itemize}
    
    
    % 
    \section{Cuerpo (Metodología, modelo matemático, etc.)}
    
    
    %\begin{itemize}
    
    En la elaboración de este proyecto, es necesario establecer que se cumplirán con los objetivos descritos anteriormente. Para el diseño y la mejora en sistemas productivos. De manera sistemática, se analizarán los procesos existentes, así como las instalaciones, herramientas, métodos, equipo y todos los factores que intervengan en este proceso de ensamblaje. Se identificarán áreas de mejora utilizando herramientas de análisis, en este caso se realizará un estudio de tiempos y movimientos. La obtención de la información y datos necesarios ,se realizó en el Instituto Tecnológico Nacional de México Campus Querétaro específicamente en el semestre enero-junio.
    
   Para el diseño, implementación y mejora de los sistemas se realizarán análisis detallados de los flujos de trabajo actuales, identificando cuellos de botella y oportunidades de mejora.Se pondrán en práctica nuevas metodologías de trabajo, como el trabajo en equipo colaborativo y la gestión ágil de proyectos.
   
    Se realizó la obtención de dos muestras continuas a partir de una cámara de video.  Continuando con el desarrollo de este proyecto, se llevaron a cabo las mejoras de propuestas y métodos utilizando metodologías ágiles, para determinar la medición del tiempo en ciclos.
    
    
    Se establecerán sistemas de retroalimentación para recopilar comentarios de los operarios que realizarán el ensamble para conocer sus incomodidades y retrasos en el proceso, así realizar los ajustes necesarios.Se aplicará el método de estudio de tiempos y movimientos para optimizar la eficiencia de las operaciones.
    %
    %
    \begin{figure}[H]
        \centering
        \includegraphics[scale=0.4]{32/img/diagramaMetodología.png}
        \caption{Diagrama de la metodología para la sección del desarrollo del sistema de clasificación.}
        \label{fig:enter-label}
    \end{figure}
    %
    %
    \begin{figure}[H]
        \centering
        \includegraphics[scale=0.5]{32/img/cadenaDeMedida.png}
        \caption{Cadena De Medida}
        \label{fig:enter-label}
    \end{figure}
    %
    %
    \begin{figure}[H]
        \centering
        \includegraphics[scale=0.4]{32/img/diagramEntradasySalidas.png}
        \caption{Diagrama de entradas y salidas para ilustrar la relación entre los componentes des sistema y cómo interactúan entre sí.}
        \label{fig:enter-label}
    \end{figure}
    
    %
    %
    \subsection{Desarrollo de la guía de plan de emergencia} Un Plan de Acción de Emergencia en el contexto industrial es un documento integral que detalla los procedimientos y responsabilidades necesarios para garantizar la seguridad y el bienestar de los empleados y la protección de la propiedad durante una emergencia. Este plan generalmente incluye instrucciones detalladas para evacuaciones, refugio en el lugar, protocolos de comunicación y coordinación con los servicios de emergencia.
    En la elaboración de esta guía se hizo una recopilación de datos, se analizaron los procesos que se llevaban a cabo en la institución en caso de emergencia. A partir de esto se realizó una clasificación de riesgos internos y externos asignando una probabilidad de ocurrencia a cada uno de ellos . Posteriormente se se realizó un programa de actividades de prevención de auxilio y se desarrolló un plan de acción, así como la identificación de capacidades. De igual manera se analizó el plano de localización de recursos, es decir, donde están localizada la señaletica o extintores de emergencia.
    Proseguimos con la identificación de riesgos externos y de puntos de reunión, tambien se desarrolló una brigada de evacuación y se estableció un directorio telefónico de emergencia.
    %
    \subsection{Análisis de los métodos, materiales, herramientas e instalación utilizada en la ejecución del ensamble de un circuito eléctrico.}

    \subsubsection{Planeación}
    
    En el proceso de elaboración de este proyecto, primeramente se planteó el objeto de estudio que se utilizaría para realizar esta experimentación, que es un circuito eléctrico conformado por una tarjeta LCD y un ESP-32.Como primer paso, nos familiarizamos con los componentes de este ensamble; los medimos, tocamos, olimos y todo lo necesario para a partir de esto, plasmarlos en algún programa asistido por computadora especificando las medias de cada uno, en este caso se utilizó Solid Works en la versión 2019.
    
     Posteriormente, se elaboró un manual en dónde se describe cada herramienta, equipo, maquina y todos los pasos que el operador debe de seguir para llegar al ensamble final. s.Visuelice el apendice \ref{anexo:manualEnsambleEléctrico.pdf}.
    
    
    \subsubsection{5´S}
    \end{itemize}
    
    \subsubsection{Desarrollo de sistemas de tiempo predeterminado}
    \subsubsection{Desarrollo de muestro de trabajo}
    \subsubsection{Datos estándar continuos y discretos}
    %
    %
    \subsection{Diseño de la forma más ecocómica de realizar el trabajo}
    %
    %
    \subsection{Normalización de métodos, materiales, herramientas e instalaciones}
    %
    %
    \subsection{Determinación del tiempo estándar para que un operador competente realice el trabajo con marcha normal}
    
    
    
    %\end{itemize}
    
    

    

    
    \subsection{Ecuaciones}
    
    Las ecuaciones son una excepción a las especificaciones prescritas de esta plantilla. 
    Deberá determinar si su ecuación debe escribirse o no utilizando la fuente Adobe Devangari. 
    Para crear ecuaciones multinivel, puede ser necesario tratar la ecuación como un gráfico e insertarla en el texto después de aplicar el estilo de la platilla.
    Las ecuaciones serán enumeradas de manera consecutiva, y el número de ecuación, entre paréntesis, se colocan al ras de la derecha, utilizando una tabulación derecha. 
    
    \begin{equation}
        \label{eq1}
        x + y = z 
    \end{equation}
    
    Es importante asegurarse de que los símbolos de la ecuación sean definidos antes o inmediatamente después de la ecuación. Utilice “(1)”, en vez de “Eq. 1” al enumerar las ecuaciones, excepto al principio de una oración: “La ecuación (\ref{eq1}) es…”
    
    \section{Resultados y discusión}
    \subsection{Desarrolo de la guía de emergemcia}
    
    Con la finalidad de eliminar todos los posibles riesgos de emergencia e incendio, se realizó eata guía de plan de emergencia. De igual manera se implementaron nuevas estrategias como es la capacitación constante del personal de como actuar en caso de una emergencia, esto con la finalidad de que en caso de ocurrencia sepan como actuarr, tomen las mejores decisiones y salvaguarden la integridad fisica de todas las personas en riesgo. 
    A continuación se describe la ubicación y datos generales donde se llevó a cabo el ensamble para posteriormente analizar lo antes planteado.
    \begin{figure}[H]
        \centering
        \includegraphics[scale=0.4]{32/img/ubicaciónDelInstituto.png}
        \caption{Av Tecnológico S/N, Centro Histórico, Centro, 76000 Santiago de Querétaro, Qro.}
        \label{fig:enter-label}
    \end{figure}
    %
    %
    \begin{figure}[H]
        \centering
        \includegraphics[scale=0.2]{32/img/croquisITQ.jpg}
        \caption{Croquis Instituto Tecnológico de Querétaro}
        \label{fig:enter-label}
    \end{figure}
    \subsubsection{Identificación del Riesgo}
    Se presenta un análisis detallado de la identificación de riesgos realizada durante el desarrollo del plan de emergencia. Este análisis incluye una evaluación sistemática de los riesgos internos y externos que pueden afectar la operatividad y seguridad de la industria. Véase el diagrama  \ref{fig:identificacion-riesgos}

    %
    %
    \begin{figure}[H]
        \centering
        \includegraphics[scale=0.25]{32/img/diagramaIdentificaciónDeRiesgos.png}
        \caption{Diagrama para la identificación de riesgos}
        \label{fig:identificacion-riesgos}
    \end{figure}
    %
    %
    \subsubsection{Riesgos internos}
    %
    %
    \subsection{Autores y Afiliaciones}
    
    Para distinguir las afiliaciones de los autores, utilice superíndices iniciando con el número 1, 2, etc., sucesivamente, esto dependerá de la cantidad de los departamentos a los que estén afiliados los autores. En caso de que todos los autores pertenezcan a una mismo departamento e institución, utilizar sólo el superíndice 1. 
    
    \subsection{Identificar los encabezados}
    
    Se les recuerda a los autores que los encabezados deben de estar conforme los solicita la guía del autor. De ahí se puede adaptar el trabajo para que sea más fácil de entender para el lector.
    Los encabezados organizan los temas sobre una base relacional y jerárquica. Por ejemplo, el título del documento es encabezado del texto principal porque todo el material posterior se relaciona y elabora sobre este tema. 
    
    \subsection{Tablas y Figuras}

    \newpage
    %%%%%%%%%%%%%%%%%%%%%%%%%%%%%%%%%%
    % 
    % 
    % \section[\appendixname~\thesection]{Apéndice}
    \centering{\section[\appendixautorefname{evidenciaEnsamble.pdf}]{}}\label{anexo:evidenciaEnsamble.pdf}
    
    \includepdf[pages=-]{32/img/evidenciaEnsamble.pdf}
    %%%%%%%%%%%%%%%%%%%%%%%%%%%%%%%%%%%%%%%%
    
    \begin{enumerate}
        \item Posición de las tablas y figuras: Coloque las figuras y las tablas en la parte superior e inferior de las columnas. Evite colocarlos en medio. Las figuras y las tablas grandes pueden abarcar ambas columnas. Los títulos de las figuras deben de estar debajo de las mismas; los títulos de las tablas deben aparecer encima de ellas. Insértese las figuras y los cuadros después de citarse en el texto. Utilice la abreviatura “Fig. 1”, incluso al principio de una oración. 
    \end{enumerate}
    
    \section{Conclusiones}
    
    Se describe aquí el alcance del trabajo, logros obtenidos y perspectivas para el futuro de este. Se sugiere colocar información cuantitativa obtenida.
    
    \section{Agradecimientos}
    
    Es importante darles su debido reconocimiento a los laboratorios, instituciones, organizaciones, entre otros que han sido participes para la culminación de este trabajo. También es importante mencionar, fondos, proyectos, becas, entre otros que se le han otorgado al o los autores para realizar el trabajo de investigación. Ejemplo: “Los autores agradecen al Concejo Nacional de Ciencia y Tecnología por los recursos otorgados…”
    
    
    

    % Ejemplo
    %  @Article{article,
    % 	author = "Author1 LastName1 and Author2 LastName2 and Author3 LastName3",
    % 	title = "Article Title",
    % 	volume = "30",
    % 	number = "30",
    % 	pages = "10127-10134",
    % 	year = "2013",
    % 	doi = "10.3389/fnins.2013.12345",
    % 	URL = "http://www.frontiersin.org/Journal/10.3389/fnins.2013.12345/abstract",
    % 	journal = "Frontiers in Neuroscience"
    % }
    
    % @book{book,
    %   author    = {Author Name}, 
    %   title     = {The title of the work},
    %   publisher = {The name of the publisher},
    %   address   = {The city},
    %   year      = 1993,
    % }
    
    % @incollection{chapter,
    %   author       = {Bauthor Surname}, 
    %   title        = {The title of the work},
    %   editor       = {Editor Name},
    %   booktitle    = {The title of the book},
    %   publisher    = {The name of the publisher},
    %   address      = {The city},
    %   year         = 2002,
    %   pages        = {201-213},
    % }
    
    % @InProceedings{conference,
    %   author = {Cauthor Name and Dauthor Surname and Fauthor LastName},
    %   title = {The title of the work},
    %   booktitle = {The title of the conference proceedings},
    %   year = 1996,
    %   publisher = {The name of the publisher},
    %   editor = {Editor Name1 and Editor Name2},
    %   pages = {41-50},
    % }
    
    % @book{cho,
    %   author       = {Gauthor Name1}, 
    %   title        = {The title of the work},
    %   publisher = {Country code and patent number},
    %   address      = {Patent Country},
    %   year = 2013
    % }
    
    % @book{patent,
    %   author    = {Hauthor Surname1}, 
    %   title     = {The title of the work},
    %   publisher = {Patent number},
    %   address   = {Patent country},
    %   year      = 2010,
    % }
    
    % % please use misc for datasets
    % @misc{dataset, 
    % 	author = "Author1 LastName1 and Author2 LastName2 and Author3 LastName3",
    % 	title = "Data Title",
    % 	year = "2011",
    % 	doi = "10.000/55555",
    % 	URL = "http://www.frontiersin.org/",
    % }
    
    \bibliographystyle{ieeetr}
    \bibliography{32/referencias}
    % 
    % 
    %%%%%%%%%%%%%%%%%%%%%%%%%%%%%%%%%%
    \appendix
    %%%%%%%%%%%%%%%%%%%%%%%%%%%%%%%%%%
    % 
    % 
    \centering{\section[\appendixautorefname{}]{APÉNDICE}}\label{anexo:pines}
    % \includepdf[pages=-]{6/Img/pines.pdf}
    
    %%%%%%%%%%%%%%%%%%%%%%%%%%%%%%%%%%%%%%%%

     
    
    %%%%%%%%%%%%%%%%%%%%%%%%%%%%%%%%%%
    \appendix
    %%%%%%%%%%%%%%%%%%%%%%%%%%%%%%%%%%
    % 
    % 
    % \section[\appendixname~\thesection]{Apéndice}
    \centering{\section[\appendixautorefname{manualEnsambleElectrico.pdf }]{}}\label{anexo:manualEnsambleEléctrico.pdf}
    
    \includepdf[pages=-]{32/img/manualEnsambleEléctrico.pdf}
    %%%%%%%%%%%%%%%%%%%%%%%%%%%%%%%%%%%%%%%%