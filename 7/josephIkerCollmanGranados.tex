\section{Joseph Iker Collman Granados}
\subsection{Definiciones}

\begin{enumerate}
    \item Precision : Es la variación, disperción, poca variación, significa un buen grado de precisión.

    \item Exactitud : Se define con respecto a su cercania (sesgo), mayor cercania implica un buen grado de exactitud.

    \item Estudio:Esfuerzo que pone el entendimiento aplicándose a conocer algo.
    \\Trabajo empleado en aprender y cultivar una ciencia o arte.
    
    \item Trabajo : Acción y efecto de trabajar.
    \\Cosa que es resultado de la actividad humana.
    
    \item Estudio de movimientos y tiempos: Es el análisis de métodos, materiales, herramientas e instalación utilizada o que se ha de utilizar en la ejecución de un trabajo.
    
    \item Analisis : Distinción y separación de las partes de algo para conocer su composición
    \\ Estudio detallado de algo, especialmente de una obra o de un escrito
    
    \item Sistema : Conjunto de reglas o principios sobre una materia racionalmente enlazados entre sí.
    \\Conjunto de cosas que relacionadas entre sí ordenadamente contribuyen a determinado objeto
    
    \item Tiempo: Duración de las cosas sujetas a mudanza.
     \\Magnitud física que permite ordenar la secuencia de los sucesos, estableciendo un pasado, un presente y un futuro, y cuya unidad en el sistema internacional es el segundo.
     
    \item Predeterminar : Determinar o resolver con anticipación algo.
    
    \item Sistema de Tiempo Predeterminado (STP):Conjunto de reglas o métodos para determinar con anticipación la secuencia de sucesos
    
    \item Estudio de Micromovimientos : Division de la asignación de trabajo en therbligs que se logra mediante el analisis, cuadro por cuadro de una pelicula y la mejora de la operación a traves de la eliminación de los movimientos innecesarios y la simplificación de los necesarios.
    
    \item Estudio de Metodos : factores fundamentales en la determinación de la productividad de los operarios
    
    \item MTM : Métodos de medición del tiempo.
    \\Medición de tiempos de métodos.
    
    \item TMU : Unidades de medida de tiempo.
    
    \item Alcanzar : Por alcanzar se entiende el movimiento realizado con la mano vacía.
    
    \item Mover :Se refiere al movimiento con un objeto en la mano.
    
    \item FD : factor dinamico
    
    \item CE-TMU : Constante Estadistica TMU. 
    
    \item Muestreo : Acción de escoger muestras representativas de la caludad o condiciones meidias de un todo.
        \\Técnica empleada en un muestreo.
        \\ Selección de una pequeña parte estadísticamente determinada, utilizada para inferir el valor de una o varias características del conjunto.
        
    \item Representativo , va : Que sirve para representar algo.
    \\Que representa con justos títulos.
    
    \item Inferir :Que representa con justos títulos.
    \\Incluir o llevar consigo algo.
    
    \item Muestreo : Acción de escoger muestras que describan de manera exacta las características de un conjunto de datos que permitirán deducir y sacar conclusiones del fenómeno a estudiar.
    
    \item Representativo:Que sirve para representar algo.
    \\ {Característico, propio, peculiar, específico, clásico, típico, individual.}
    
    \item Inferir :Deducir algo o sacarlo como conclusión de otra cosa.
       \\Incluir o llevar consigo algo
       
    \item Muestreo del trabajo:Herramienta para disminuir el costo que se presenta en el estudio continuo del tiempo.
    
    \item Muestreo Discreto :Implica seleccionar elementos específicos de una población finita o contable, donde cada elemento tiene una probabilidad asociada de ser elegido.
    
    \item Muestreo Continuo:Selección de elementos de una población infinita, donde los valores pueden tomar cualquier valor dentro de un rango específico, como la altura o el tiempo.

    \item Estudio de tiempos Convencional :Es una muestra continua de n ciclos (Suponiendo que la distribución estadística es normal).
    
    \item Distribucion estadistica normal :Forma de distribución estadística simétrica con forma de campana, donde la mayoría de los datos se concentran cerca de la media.

    \item Estudio de tiempos no convencional :Es una muestra discreta (Suponiendo que la distribución estadística es binomial).
    
    \item Distribucion estadistica Binomail : estadístico que describe la probabilidad de obtener un número específico de éxitos en un número fijo de ensayos independientes, donde cada ensayo tiene dos resultados posibles: éxito o fracaso.

    \item Pronostico:Valor que se cree obtener.
    
    \item Estimacion : Analisis a traves de operaciones.

\end{enumerate}
\bibliographystyle{apalike}
\bibliography{7/referencias}
