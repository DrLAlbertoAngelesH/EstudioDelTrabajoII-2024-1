\section{José Natividad Mireles Navarro}
\subsection{Definiciones}



\begin{enumerate}
    \item Muestreo : Acción de escoger muestras representativas de la calidad o condiciones medias de un todo.
    \item Muestra : Porción de un producto o mercancía que sirve para conocer la calidad del género.
    \item Tiempo ;  Magnitud física que permite ordenar la secuencia de los sucesos, estableciendo un pasado, un presente y un futuro, y cuya unidad en el sistema internacional es el segundo.
    \item Eficiencia: Capacidad de lograr los resultados deseados con el mínimo posible de recursos.
    \item Optimizar: Buscar la mejor manera de realizar una actividad.
    \item Movimiento: Estado de los cuerpos mientras cambian de lugar o de posición.
    \item Estudio de movimiento:Eliminar o mejorar elementos innecesarios que podrían afectar la productividad, seguridad, y calidad de la producción
    \item Fatiga: Molestia ocasionada por un esfuerzo más o menos prolongado o por otras causas, y que en ocasiones produce alteraciones físicas.
    \item Método:Modo de obrar o proceder, hábito o costumbre que cada uno tiene y observa.
    \item Modo: Procedimiento o conjunto de procedimientos para realizar una acción \cite{RAE}.
\end{enumerate}