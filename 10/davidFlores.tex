\section{David Ricardo Flores Antonio}
\subsection{Definiciones}

\begin{enumerate}
    \item \textbf{Estudio:} Esfuerzo que pone el entendimiento aplicandose a conocer algo.
    \item \textbf{Trabajo:} Cosa que es resultado de la actividad humana.
    \item \textbf{Estudio de movimientos y tiempos:} Es el análisis de métodos, materiales, herramientas e instalación utilizada o que se ha de utilizar en la ejecución de un trabajo.
    \item \textbf{Análisis:} Estudio detallado de algo, especialmente de una obre o un escrito.
    \item \textbf{Sistema:} Conjunto de reglas o principios sobre una materia racionalmente enlazados entre sí.
    \item \textbf{Tiempo:} Duración de las cosas sujetas a mudanza.
    \item \textbf{Predeterminar:} Determinar o resolver con anticipación algo.
    \item \textbf{Sistemas de tiempo predeterminado(STP):} Conjunto de reglas o métodos para determinar con anticipación la secuencia de sucesos.
    \item \textbf{Therbligs:} son los diecisiete movimientos en los que se puede subdividir cualquier tarea laboral para estudiar la productividad motriz de un operador en su estación de trabajo.
    \item \textbf{Muestreo:} Acción de escoger muestras que describan de manera exactas las características de un conjunto de datos que permitirán deducir y sacar conclusiones del fenómeno a estudiar.
    \item \textbf{Inferir:} Deducir algo o sacarlo como conclusión de otra cosa.
\end{enumerate}