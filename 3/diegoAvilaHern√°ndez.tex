\section{Diego Avila Hernández}
\subsection{Definiciones}

\begin{enumerate}
    \item Exactitud: Se define con respecto a su cercanía.
    \item Precisión: Es la variación o dispersión.
    \item Estudio de movimientos y tiempos: Es el análisis de métodos, materiales, herramientas e instalaciones.
    \item Análisis: Distinción y separación de las partes de algo para conocer su composición.
    \item Sistema de tiempo predeterminado: Conjunto de reglas o métodos para determinar con anticipación la secuencia de sucesos.
    \item Muestreo: Acción de escoger muestras que describan de manera exacta las características de un conjunto de datos que permitiran deducir y sacar conclusiones del fenómeno a estudiar.
    \item Ciclo útil: Es la relación que existe entre el tiempo en que la señal se encuentra en estado activo y el periodo de la misma.
    \item Tiempo estándar: Es el tiempo requerido para que un operario de tiempo medio, plenamente calificado, adiestrado y trabajando a un ritmo normal lleve a cabo la operación.
    \item Tiempo base máquina: Es el tiempo que la maquina funciona automáticamente y no necesita la dirección, ni ningún tipo de ayuda de un operario.
    \item Factor de tolerancia: Se refiere al límite establecido para la cantidad máxima de una inversión.
    
    
\end{enumerate}