\section{Pilar Martinez Oscar}
\subsection{Definiciones}

\begin{enumerate}
    \item Exactitud: 
    Se definecon repecto a su cercania(sesgo).
    \item Precision: 
    Es la variacion o dispersion.
    \item Estudio de movimientos y tiemopos:
    Analisis de metodos, herramienmtas e instalaciones. 
    \item Analisis: 
    Distincion y separacion de las partes de algo para conocerr su composicion. 
    \item Sistemas de tiempo predeterminado: 
    Conjunto de reglas o metodos para determinar con anticipacion la secuencia de sucesos.
    \item Muestreo: 
    Accion de escoger muestras que describan de manera exacta las caracteristicas de un  conjunto de datos que permitan deducir y obtener conclusiones del fenomeno a estudiar.
    \item Ciclo util: 
    Es la relación que existe entre el tiempo en que la señal se encuentra en estado activo y el periodo de la misma.
    \item Tiempo estandar: 
    Es el tiempo requerido para que un operario de tiempo medio, plenamente calificado, adiestrado y trabajando a un ritmo normal lleve a cabo la operación.
    \item Tiempo base maquina: 
    Es el tiempo que la maquina funciona automáticamente y no necesita la dirección, ni ningún tipo de ayuda de un operario.
    \item Factor de tolerancia: 
    Se refiere al límite establecido para la cantidad máxima de una inversión.

    
\end{enumerate}