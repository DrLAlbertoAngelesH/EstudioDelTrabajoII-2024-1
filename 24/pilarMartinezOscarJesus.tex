\lhead{\begin{tikzpicture}[remember picture, overlay]
    \node [anchor=100,inner sep=0] (imagenIZQUIERDA) at (current page header area.north){\includegraphics[width=18cm]{img/Encabezado.PNG}};
    \end{tikzpicture}}
    \rhead{Pilar-Martinez}
    \rfoot{\begin{tikzpicture}[remember picture, overlay]
    \node [anchor=140,inner sep=0] (imagenDERECHA) at (current page footer area.south){\includegraphics[width=18cm]{img/Foot.PNG}};
    \end{tikzpicture}}
    %----------------------------------------------------------------------------------------
    \lfoot{ \thepage}
    % \renewcommand{\labelenumi}{\alph{enumi}.)} 
    %----------------------------------------------------------------------------------------
    %----------------------------------------------------------------------------------------
    %	TITLE SECTION
    %----------------------------------------------------------------------------------------
    
    \setlength{\droptitle}{-5\baselineskip} % Move the title up
    \title{\textbf{Estudio de tiempos y movimientos en el ensamble de un circuito electrónico utilizando diferentes métodos para su optimización }} % Article title
    
     \author{ 
     \textsc{Pilar Martinez, Oscar Jesús}\\ 
    %  Afiliación:
     \texttt{ Nombre Instituto } \\ 
     \texttt{Nombre de la Organización } \\ 
     \texttt{Ciudad y País}\\ 
     \texttt{Correo} 
     \and 
     \textsc{Ángeles-Hurtado, Luis Alberto}\\ 
    %  Afiliación:
     \texttt{ Instituto Tecnológico de Querétaro } \\ 
     \texttt{ Tecnológico Nacional de México } \\ 
     \texttt{Querétaro, México}\\ 
     \texttt{alb3rt0.ah@gmail.com} 
    }
    
    
    %----------------------------------------------------------------------------------------
    
    % \begin{document}
    
    % Print the title
    \maketitle
    \thispagestyle{fancy}
    
    %----------------------------------------------------------------------------------------
    %	ARTICLE CONTENTS
    %----------------------------------------------------------------------------------------
    
    % \section*{Resumen}
    % \textit{Palabras clave:}
    % El resumen (ancho de página) deberá contener entre 100 y 200 palabras tipo Adobe Devangari 11 puntos.
    
    \begin{abstract}
    \noindent 
    El resumen (ancho de página) deberá contener entre 100 y 200 palabras tipo Adobe Devangari 11 puntos.
    
    \end{abstract}
    % 
    % 
    \textbf{\textit{Palabras clave}}: {First keyword should be the corresponding to the research area according with the authors guide. Maximum of 6 keywords.}
    % \keywords{First keyword should be the corresponding to the research area according with the authors guide. Maximum of 6 keywords.}
    
    \section{Introducción}
    
    % Define estudio de tiempos y movimientos
    % define que es ensamble
    % define que es circuito electronico
    % define el metodo de tiempos predeterminados
    \begin{itemize}
        \item Estudio de tiempos y movimientos: Estudio
        1. m. Esfuerzo que pone el entendimiento aplicándose a conocer algo.
        2. m. Trabajo empleado en aprender y cultivar una ciencia o arte.
       Tiempo
        2. m. Magnitud física que permite ordenar la secuencia de los sucesos, estableciendo un pasado, un presente y un futuro, y cuya unidad en el sistema internacional es el segundo.  Movimiento
        2. m. Estado de los cuerpos mientras cambian de lugar o de posición.
        \item "El estudio de tiempos y movimientos es una tecnica de medición de las actividades, condiciones, herramientas y materiales efectuados para  realizar cambio de posición en los materiales"
        \item Ensamble:
        "Proceso de ensamble implica la colocación de dos o más piezas individuales para la conformación de un producto final"
        \item Circuito electronico: Es la unión de dos o más elementos que al conectarse permiten el flujo de la corriente eléctrica. Este mecanismo facilita y a la vez controla el paso de la electricidad; es posible que esté formado por diferentes elementos encargados de determinar sus características, algunos de estos son: fuentes, interruptores, resistencias, condensadores, semiconductores, cables, entre otros.
        \item Metodo de tiempos predeterminados: El método de tiempos predeterminados es una técnica de medición del trabajo que utiliza estimaciones predefinidas de cuánto tiempo tomará completar una tarea. Se basa en estándares establecidos para diferentes acciones dentro de la tarea en lugar de medir el tiempo real. Es útil para planificar y programar tareas en entornos industriales donde la precisión en la estimación del tiempo es esencial para la eficiencia.
    \end{itemize}
    % 
    % 
    \section{Justificación}
    
    \begin{itemize}
        \item Buscamos encontrar la mejora continua en los tiempos y movimientos que llevamos acabo en el ensamble, dando una mayor cobertura a las intrucciones que lleva acabo el operador para lograrlo,buscando la eficazia seguido de la utilizacion correcta de los recursos proporcionados, explicando y comprendiendo el conocimiento de quien realizara el ensamble. 
        Comprendiendo que en la actualidad las empresas demandan un alto nivel de conocimiento tanto a nivel productivo como para ejecutar y seguir instrucciones. 
        Para lograr este nivel de entendimiento tanto en los operadores, como en quien realiza la optimizacion, debemos apoyarnos en los programas, tecnicas y conocimientos brindados a lo largo de nuestra enseñanza profesional. teniendo muy en cuenta que siempre existira la manera de mejorar los procesos y minimizar el esfuerzo, para reducir los errores, aumnetando la calidad del producto al que deseamos llegar. 
    \end{itemize}
    % 
    % 
    \section{Descripción del problema}
    \begin{itemize}
        \item Se busca la mejor manera de lograr optimizar, integrar y analizar los tiempos y movimientos para obtener el ensable del circuito electronico, con la mayor reduccion de tiempo posibles, utilizando los materiales a nuestro alcance, como lo son la planificacion, los planos, tablas y la asociacion con los materiales, para asi conseguir el objetivo, garantizando al operador la capacidad de la estandarizacion de sus tiempos y movimientos.
    \end{itemize}
    
    \textbf{*La incógnita científica es el elemento cuya solución incrementa el conocimiento
    científico.}
    % 
    % 
    \section{Fundamentación teórica}
    
    Se utilizaran metodos con el fin de optimizar procesos que conyeben a normas para una mayor eficazia en cada uno de los procesos, teniendo precente la forma en que se lleva acabao el trabajo, para una optimizacion en el tiempo y movimientos ocupados, Conociendo que esto proporciona la base para mejorar los procesos de produccion  y todo lo relacionado a tiempos y movimientos. En esta actividad se busca mejorar el tiempo de ensamble, defininendo y conociendo pasos sistematicos, dando la mayor utilidad posible a los materiales y centrandonos en definir los movimientos innecesarios, eliminandolos o acortantolos para una mayor efectividad. Teniendo en cuenta que buscamos "La forma mas economica de realizar el trabajo", lo que nos llevara a una mayor productividad y reduccion de costos de produccion. 
    
    \begin{itemize}
        \item Se debe de retomar el tema que se planteo en la introducción, pero ahora profundizando para clarificar la incógnita científica y se pueda plantear la hipótesis.
        \item Se debe de retomar la descripción del problema, pero ahora a profundidad del (los) objeto(s) de estudio. 
        \item Se debe de profundizar en las metodologías que se ha usado para el estudio del tema.
        \item Referencias solo de artículos y libros científicos.
    \end{itemize}
    % 
    % 
    \section{Hipótesis}
    
    Es la suposición con fundamento científico relativa a la solución del problema, necesidad o de cómo se aprovecha la oportunidad con la incógnita científica y se fundamenta con: 1. Una suposición (en afirmativo o negativo) y ésta deberá vincularse con:
    2. La fundamentación científica que deberá ser precisa 3. Una entidad de comparación para probar la suposición y
    4. La variable con que se califica o cuantifica la comparación o se prueba la hipótesis.
    
    \begin{itemize}
        \item Se debe de identificar claramente la suposición científica
        \item Se debe de identificar claramente el fundamento científico
        \item Se debe identificar claramente la variable de respuesta
        \item Se debe identifican claramente las realidades o modelos contrastantes
        \item Se debe de establecer las variables asociadas, explicativas o que tienen relación funcional con la variable de respuesta
    \end{itemize}
    % 
    % 
    \section{Objetivo}
    
    Precisar la acción necesaria para probar la hipótesis. Dicha acción se establece mediante el uso de verbos activos y en infinitivo.
    \begin{itemize}
        \item Se debe establecer que se pretende probar la hipótesis
    \end{itemize}
    
    \subsection{Objetivos específicos }
    
    \begin{itemize}
        \item Se debe establecer como un conjunto de acciones comunes para lograr el objetivo general
        \item Se debe establecer como etapas para lograr el objetivo general
    \end{itemize}
    
    Son actividades orientadas al cumplimiento del objetivo general. Se establecen con verbos activos en infinitivo. Son parte de la acción encaminada a probar la hipótesis. Éstos deben ser precisos, y en lo posible evitar aspectos metodológicos.
    % 
    % 
    \section{Cuerpo (Metodología, modelo matemático, etc.)}
    % 
    % 
    \begin{figure}[H]
        \centering
        \includegraphics[trim = {20mm 80mm 25mm 160mm},clip,scale=0.5]{24/Img/Ensamble Circuito Electronico.pdf}
        \caption{Ensamble Circuito Electronico}
        \label{fig:Ensamble Circuito Electronico}
    \end{figure}
    % 
    %
    % 
    % 
    \begin{figure}[H]
        \centering
        \includegraphics[trim = {0mm 0mm 0mm 0mm},clip,scale=0.350]{24/Img/CIRCUITO ELECTRICO (1).pdf}
        \caption{CIRCUITO ELECTRICO METODOLOGIA}
        \label{fig:CIRCUITO ELECTRICO }
    \end{figure}
    % 
    %
    % 
    % 
    \begin{figure}[H]
        \centering
        \includegraphics[trim = {0mm 180mm 0mm 0mm},clip,scale=0.350]{24/Img/CIRCUITO ELECTRICO (2).pdf}
        \caption{CIRCUITO ELECTRICO METODOLOGIA}
        \label{fig:CIRCUITO ELECTRICO }
    \end{figure}
    % 
    %
    %
    %
    \begin{figure}[H]
        \centering
        \includegraphics[trim = {10mm 10mm 10mm 10mm},clip,scale=0.120]{24/Img/Cable de conexion MH.PDF}
        \caption{Cable de conexion MH}
        \label{fig:Cable de conexion MH}
    \end{figure}
    % 
    %
    % 
    % 
    \begin{figure}[H]
        \centering
        \includegraphics[trim = {10mm 10mm 10mm 10mm},clip,scale=0.120]{24/Img/Cable de conexion MM.pdf}
        \caption{Cable de conexion MM}
        \label{fig:Cable de conexion MM}
    \end{figure}
    % 
    %
    % 
    % 
    \begin{figure}[H]
        \centering
        \includegraphics[trim = {10mm 10mm 10mm 10mm},clip,scale=0.120]{24/Img/Potenciometro.pdf}
        \caption{Potenciometro}
        \label{fig:Potenciometro}
    \end{figure}
    % 
    %
    % 
    % 
    \begin{figure}[H]
        \centering
        \includegraphics[trim = {10mm 10mm 10mm 10mm},clip,scale=0.120]{24/Img/LCD 16X2.pdf}
        \caption{LCD 16X2}
        \label{fig:LCD 16X2}
    \end{figure}
    % 
    %% 
    % 
    \begin{figure}[H]
        \centering
        \includegraphics[trim = {10mm 10mm 10mm 10mm},clip,scale=0.120]{24/Img/Modulo 12C.pdf}
        \caption{Modulo 12C}
        \label{fig:Modulo 12C}
    \end{figure}
    % 
    %
    % 
    % 
    \begin{figure}[H]
        \centering
        \includegraphics[trim = {10mm 10mm 10mm 10mm},clip,scale=0.120]{24/Img/ESP32.pdf}
        \caption{ESP32}
        \label{fig:ESP32}
    \end{figure}
    % 
    %
    % 
    % 
    \begin{figure}[H]
        \centering
        \includegraphics[trim = {10mm 10mm 10mm 10mm},clip,scale=0.120]{24/Img/Resistencia.pdf}
        \caption{Resistencia}
        \label{fig:Resistencia}
    %
    %
    \end{figure}
    % 
    %
    Cada estrategia metodológica se establece acorde a cada objetivo, y por tanto deberá ser desglosada precisada y ordenada claramente. En consecuencia cada objetivo que se presentó en forma de verbo en infinitivo deberá determinar una estrategia en forma de adverbio. Ej. Desarrollar…Desarrollo. Son las actividades ordenadas que tienen como finalidad la prueba de la hipótesis. 
    
    \begin{itemize}
        \item Se debe establecer que se habrá de hacer, como, conque, y donde para obtener la información que permita probar la hipótesis.  
        \item Se debe desglosar de acuerdo a los objetivos específicos. 
        \item Se debe establecer una estrategia metodológica por cada objetivo específico. De manera simplista se podría decir que se cambia el verbo en infinitivo por su respectivo adverbio.
        \item En cada objetivo se debe describir que método, que materiales y que equipo se usará para conseguirlo.
        \item Se deben tener referencias.
    \end{itemize}
    % 
    % 
    \begin{figure} [H]
    
    \end{figure}
    
    %
    \subsection{Prepara tu documento}
    
    Antes de que comiences a utilizar esta plantilla, es recomendable que prepare la información que contendrá en un archivo aparte. 
    Ten preparadas tus gráficas, así como también las tablas aparte, para que sea más fácil integrarlo. 
    Se recomienda fuertemente el uso de \textbf{formato Enhanced Metafile (.emf) para imágenes y gráficas} de resolución óptima. 
    Finalmente, completa y organiza el contenido antes de darle el formato de esta plantilla. 
    
    \subsection{Acrónimos y Abreviaciones}
    
    Los acrónimos y abreviaciones deberán ser definidos únicamente la primera vez que aparecen en el texto, esto para que el lector entienda lo que significan.
    
    \subsection{Ecuaciones}
    
    Las ecuaciones son una excepción a las especificaciones prescritas de esta plantilla. 
    Deberá determinar si su ecuación debe escribirse o no utilizando la fuente Adobe Devangari. 
    Para crear ecuaciones multinivel, puede ser necesario tratar la ecuación como un gráfico e insertarla en el texto después de aplicar el estilo de la platilla.
    Las ecuaciones serán enumeradas de manera consecutiva, y el número de ecuación, entre paréntesis, se colocan al ras de la derecha, utilizando una tabulación derecha. 
    
    \begin{equation}
        \label{eq1}
        x + y = z 
    \end{equation}
    
    Es importante asegurarse de que los símbolos de la ecuación sean definidos antes o inmediatamente después de la ecuación. Utilice “(1)”, en vez de “Eq. 1” al enumerar las ecuaciones, excepto al principio de una oración: “La ecuación (\ref{eq1}) es…”
    
    \section{Resultados y discusión}
    
    Antes de comenzar a preparar tu artículo, es importante que lea primero la guía del autor, la cual incluye los temas o apartados que son necesarios para tener tu trabajo completo.
    Una vez completada la edición del texto, el documento está listo para el uso de esta plantilla. En este archivo recién creado, resalte todo el contenido e importe el archivo de texto preparado. Ahora esta listo para estilizar su documento.
    En esta sección se deben presentar todo lo obtenido de la sección 2, incluidas deducciones o efectos del desarrollo. También se podrán incluir subsecciones numeradas de la siguiente forma:
    
    \subsection{Autores y Afiliaciones}
    
    Para distinguir las afiliaciones de los autores, utilice superíndices iniciando con el número 1, 2, etc., sucesivamente, esto dependerá de la cantidad de los departamentos a los que estén afiliados los autores. En caso de que todos los autores pertenezcan a una mismo departamento e institución, utilizar sólo el superíndice 1. 
    
    \subsection{Identificar los encabezados}
    
    Se les recuerda a los autores que los encabezados deben de estar conforme los solicita la guía del autor. De ahí se puede adaptar el trabajo para que sea más fácil de entender para el lector.
    Los encabezados organizan los temas sobre una base relacional y jerárquica. Por ejemplo, el título del documento es encabezado del texto principal porque todo el material posterior se relaciona y elabora sobre este tema. 
    
    \subsection{Tablas y Figuras}
    
    \begin{enumerate}
        \item Posición de las tablas y figuras: Coloque las figuras y las tablas en la parte superior e inferior de las columnas. Evite colocarlos en medio. Las figuras y las tablas grandes pueden abarcar ambas columnas. Los títulos de las figuras deben de estar debajo de las mismas; los títulos de las tablas deben aparecer encima de ellas. Insértese las figuras y los cuadros después de citarse en el texto. Utilice la abreviatura “Fig. 1”, incluso al principio de una oración. 
    \end{enumerate}
    
    \section{Conclusiones}
    
    Se describe aquí el alcance del trabajo, logros obtenidos y perspectivas para el futuro de este. Se sugiere colocar información cuantitativa obtenida.
    
    \section{Agradecimientos}
    
    Es importante darles su debido reconocimiento a los laboratorios, instituciones, organizaciones, entre otros que han sido participes para la culminación de este trabajo. También es importante mencionar, fondos, proyectos, becas, entre otros que se le han otorgado al o los autores para realizar el trabajo de investigación. Ejemplo: “Los autores agradecen al Concejo Nacional de Ciencia y Tecnología por los recursos otorgados…”
    
    \section*{Referencias}
    
    Para esta platilla, se solicita al autor enumerar las citas de manera consecutiva entre corchetes \cite{YLi2013}. 
    La puntuación de la oración que sigues sería \cite{Mesaelides2011}. 
    Refiérase simplemente al número de referencia, como en \cite{Morales2012}, no utilice “Ref. [3]” o “referencia [3]” excepto al principio de una oración: “La referencia [3] fue la primera…”
    Enumere las notas al pie por separado en superíndices. Coloque la nota de pie de en la parte inferior de la columna en la que se citó. No coloque notas al pie en la lista de referencias. Utilice letras para las notas al pie de la tabla.
    A menos de que haya tres autores o más; no utilice “et al.”. Los trabajos que no hayan sido publicados, incluso si han sido presentados para su publicación, deben ser citados como “inéditos”. Los trabajos que han sido aceptados para su publicación deben de citarse como “en prensa”. Poner en mayúscula sólo la primera palabra de un título, excepto los nombres propios y los símbolos de elemento. 
    
    Otros ejemplos \cite{LAAngeles2021}, \cite{LAAngelesConni}. 
    
    % Ejemplo
    %  @Article{article,
    % 	author = "Author1 LastName1 and Author2 LastName2 and Author3 LastName3",
    % 	title = "Article Title",
    % 	volume = "30",
    % 	number = "30",
    % 	pages = "10127-10134",
    % 	year = "2013",
    % 	doi = "10.3389/fnins.2013.12345",
    % 	URL = "http://www.frontiersin.org/Journal/10.3389/fnins.2013.12345/abstract",
    % 	journal = "Frontiers in Neuroscience"
    % }
    
    % @book{book,
    %   author    = {Author Name}, 
    %   title     = {The title of the work},
    %   publisher = {The name of the publisher},
    %   address   = {The city},
    %   year      = 1993,
    % }
    
    % @incollection{chapter,
    %   author       = {Bauthor Surname}, 
    %   title        = {The title of the work},
    %   editor       = {Editor Name},
    %   booktitle    = {The title of the book},
    %   publisher    = {The name of the publisher},
    %   address      = {The city},
    %   year         = 2002,
    %   pages        = {201-213},
    % }
    
    % @InProceedings{conference,
    %   author = {Cauthor Name and Dauthor Surname and Fauthor LastName},
    %   title = {The title of the work},
    %   booktitle = {The title of the conference proceedings},
    %   year = 1996,
    %   publisher = {The name of the publisher},
    %   editor = {Editor Name1 and Editor Name2},
    %   pages = {41-50},
    % }
    
    % @book{cho,
    %   author       = {Gauthor Name1}, 
    %   title        = {The title of the work},
    %   publisher = {Country code and patent number},
    %   address      = {Patent Country},
    %   year = 2013
    % }
    
    % @book{patent,
    %   author    = {Hauthor Surname1}, 
    %   title     = {The title of the work},
    %   publisher = {Patent number},
    %   address   = {Patent country},
    %   year      = 2010,
    % }
    
    % % please use misc for datasets
    % @misc{dataset, 
    % 	author = "Author1 LastName1 and Author2 LastName2 and Author3 LastName3",
    % 	title = "Data Title",
    % 	year = "2011",
    % 	doi = "10.000/55555",
    % 	URL = "http://www.frontiersin.org/",
    % }
    
    \bibliographystyle{ieeetr}
    \bibliography{24/referencias}