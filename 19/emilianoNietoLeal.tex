\section{Emiliano Nieto Leal}
\subsection{Definiciones:}

\begin{enumerate}
    \item Estudio del trabajo:
        \begin{enumerate}   
           \item Estudio: Esfuerzo que pone el entendimiento aplicandose a conocer algo.     \cite{RAE}
           \\-Segunda Definición:Trabajo empleado en aprender y cultivar una ciencia o arte. \cite{RAE}
           \item Trabajo: Esfuerzo humano aplicado a la producción de riqueza, en contraposición a capital. \cite{RAE}
           \\-Segunda Definición: Obra, resultado de la actividad humana. \cite{RAE}
    \end{enumerate}
    \item Estudio de tiempos y movimientos:
        \begin{enumerate}   
           \item Tiempo: Duracion de las cosas sujetas a mudanza. \cite{RAE}
           \\-Segunda Definición: Magnitud física que permite ordenar la secuencia de los sucesos, estableciendo un pasado, un presente y un futuro, medida por segundo. \cite{RAE}
           \item Movimiento: Estado de los cuerpos mientras cambian de lugar o de posición. \cite{RAE}
           \item Estudio de tiempos y movimientos: Es el análisis de métodos, materiales, herramientas e instalación o que se ha de utilizar en la ejecución de un trabajo. \cite{RAE}
           \begin{enumerate}   
            \item Análisis: Distinción y separación de las partes de algo para conocer su composición. \cite{RAE}
            \\-Segunda Definición: Estudio detallado de algo, especialmente de una obra o de un escrito.  \cite{RAE}
            \end{enumerate}
    \end{enumerate}
    \item Muestreo del trabajo:
        \begin{enumerate}   
           \item Muestreo: Acción de escoger muestras representativas de la calidad o condiciones medias de un todo.  \cite{RAE}
           \item Técnica: Conjunto de procedimientos y recursos de que sirve una ciencia y un arte.  \cite{RAE}
           \item Representativo: Sirve para representar algo. \cite{RAE}
           \item Inferir: Deducir algo o sacarlo como conclusión de otra cosa.
           \cite{RAE}
    \end{enumerate}
    \item Látex: Sistema de creación de textos, orientado en técnicos y científicos para los que un procesador de textos común, no es suficiente. Facilitando mucho la escritura de fórmulas y otro tipo de elementos necesarios para estos trabajos.\cite{article{
     author = "Gobierno de Canarias",
	   title = "Herramienta: Latex",
	   year = "2015, 29 de septiembre",
	   URL = "https://www3.gobiernodecanarias.org/medusa/ecoescuela/recursosdigitales/2015/09/29/herrramienta-latex/",
      }}
    \item Látex: Sistema de creación de textos, orientado en técnicos y científicos para los que un procesador de textos común, no es suficiente. Facilitando mucho la escritura de fórmulas y otro tipo de elementos necesarios para estos trabajos.\cite{article{
       author = "Gobierno de Canarias",
	   title = "Herramienta: Latex",
	   year = "2015, 29 de septiembre",
	   URL = "https://www3.gobiernodecanarias.org/medusa/ecoescuela/recursosdigitales/2015/09/29/herrramienta-latex/",
      }}
\end{enumerate}

\newpage
\bibliographystyle{apalike}
\bibliography{19/referencias}
