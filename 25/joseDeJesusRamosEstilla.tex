\section{Jose de Jesus Ramos Estilla}
\subsection{Definiciones}

\begin{enumerate}
    \item Preciso: Se denomina precisión a la capacidad de un instrumento de dar el mismo resultado en mediciones diferentes realizadas en las mismas condiciones o de dar el resultado deseado con exactitud. Esta cualidad debe evaluarse a corto plazo. No debe confundirse con exactitud ni con reproducibilidad. 
    \cite{de_2004}
    
    \item Exacto: Igual o que se asemeja en un grado muy alto a algo o alguien que es tomado como modelo.
    \cite{asale_rae_2023}

    
    \item Estudio de movimientos y tiempos: Actividad que implica la técnica de establecer un estándar de tiempo permisible para realizar una tarea determinada, con base en la medición del contenido del trabajo del método prescrito, con la debida consideración de la fatiga y las demoras personales y los retrasos inevitables al igual del análisis cuidadoso de los diversos movimientos que efectúa el cuerpo al ejecutar un trabajo.  
    \cite{lópez_2020}

    \item Analisis: Distinción y separación de las partes de un todo hasta llegar a conocer sus principios o elementos.
     \cite{RAE}

    \item Alcanzar: Llegar a juntarse con una persona o cosa que va delante.
    \cite{soto_2022}

    \item Mover: Hacer que un cuerpo deje el lugar o espacio que ocupa y pase a ocupar otro.
    \cite{rae_rae_2020}

    \item Therblings: Al hablar de lo que son los therbligs se hace referencia a 18 movimientos en los que se puede dividir cualquier tarea que se realice, mayormente en el campo laboral pues busca determinar la productividad de un operador en la estación de trabajo donde se desarrolla.
    \cite{soto_2022}


    \item Muestreo: Acción de escoger muestras representativas de la calidad o condiciones medias de un todo.
    \cite{westreicher_2021}


    \item Representativo: Que representa o sirve para representar a alguien o algo.
    \cite{representativo_definicion_babla_2024}

    \item Inferir: Deducir algo o sacarlo como conclusión de otra cosa.
    \cite{asale_rae_2023}

    


    

    

     

\end{enumerate}
\bibliographystyle{apalike}
\bibliography{25/referencias}