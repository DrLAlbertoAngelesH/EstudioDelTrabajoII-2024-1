\section{Mitzi Daniela Vazquez Montes}
\subsection{Definiciones}
\begin{enumerate}
    \item Método de regreso a cero: El cronómetro se lee a la terminación de cada elemento, y luego se regresa a cero de inmediato. Al iniciarse el siguiente elemento el cronómetro parte de cero.
    \item Método continuo: Presenta un registro completo de todo el periodo de observación; como resultado, complace al operario y al sindicato.
    \item Muestreo de trabajo: Es una herramienta para disminuir el costo que se presenta en el estudio continuo de tiempo.
    \item MTM: (Medición de tiempos método) Es un procedimiento que analiza cualquier operación manual o método por los movimientos básicos necesarios para ejecutarlos, asignando a cada movimiento un tiempo tipo predeterminado que se define por la índole del movimiento y las condiciones en que se efectúa.
    \item Estudio de movimientos y tiempos: Es el análisis de métodos, materiales, herramientas e instalación utilizada o que se ha de utilizar en la ejecición de un trabajo.
    \item STP: (Sistema de tiempos predeterminados) Conjunto de reglas o métodos para determinar con anticipación la secuencia de sucesos. Determinar el tiempo necesario de hacer el trabajo. Están basados en los THERBLIGS.
    \item Therbligs: Son los diecisiete movimientos en los que se puede subdividir cualquier tarea laboral para estudiar la productividad motriz de un operador en su estación de trabajo \cite{book}.
\end{enumerate}

