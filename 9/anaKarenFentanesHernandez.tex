\section{Ana Karen Fentanes Hernandez}
\subsection{Definiciones}

\begin{enumerate}
    \item Estudio de movimientos y tiempos: Es analisis de materiales, herramientas en instalacion utilizada en la ejecucuion de un trabajo
    \item Latex: Es un programa donde se suben documentos base a codigos
    \item Gihtub: Repositorio, creador de codigos de programas de ordenador
    \cite{freecodecamp-2021}
    \item Git: Proyecto de codigo abierto maduro y con mantenimiento activo
    \item Preciso: En una variacion o dispercion, poca variacion significa un buen grado de precision
    \item Exactitud: Se define con respecto a su cercania , mayor cercania implica un buen grado de exactitud 
    \item Estudio: Esfuerzo que pone el entendimiento aplicandose o conoce algo
    \item Trabajo: Accion y efecto de trabajar
    \item Analisis: Definicion y separacion de las partes de algo para conocer su composicion
    \item Sistema: Conjunto de reglas o principios sobre unmateria nacionalmente enlazados entre si
    \item Tiempo: Duracion de las cosas sujetas a mudanza
    \item Duracion: Magnitud fisica que permite ordenar la secuencia de los sucesos, estableciendo un pasado, un presente y un futuro y cuya unidad en el sistema internacional es el segundo
    \item Predeterminar: Determinar o resolver con anticipacion algo
    \cite{Diccionario-02-9}
    \item Sistema de tiempos predeterminados: Conjunto de reglas o metodos para determinar con anticipacion la secuencia de sucesos
    \cite{UPIICSA-2021}
    \item Alcanzar: Por alcanzar se entiende el movimiento realizado con la mano vacia
    \item Mover: Se refiere al movimiento con un objeto en la mano
    \item Muestreo: Accion de escoger muestras representativas de la cantidad o condiciones medias de un todo
    \item Representativo: Sirve para representar algo, caracteristico, propio, peculiar, especifico, tipico
    \item Interferir: Dividir algo o sacarlo como conclusion de otra cosa, deducir. derivar, concluir
    \item El estudio continuo de tiempo: Ofrece un cuadro completo de la situacion mientras se le estudia.
    \cite{Unidad1-06-12}  
\end{enumerate}
\bibliographystyle{apalike}
\bibliography{9/referencias}