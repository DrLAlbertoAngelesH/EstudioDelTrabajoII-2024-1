\section{Cristian Jonathan}
\subsection{Definiciones}

\begin{enumerate}

    \item \textbf {Trabajo:}
    \\Acción y efecto de trabajar.
    

 
 
    \item  \textbf {Funcion: }
    \\Sirve para descubrir fenomenos de variacion y cambio.
    \item \textbf  {Estudio: }
    \\Esfuerzo que pone el entendimiento aplicándose a conocer algo o   trabajo empleado en aprender y cultivar una ciencia o arte.
    
    \item  \textbf {Exactitud: }
    \\Se define con respecto a su cercania (sesgo), entre mayor cercania implica un buen grado de exactitud.  
    
    \item  \textbf {Preciso: }
    \\Es la variacion o dispercion en la cual poca variacion implica un buen grado de precision.
    
    
    \item \textbf  {Tiempo:}
    \\Duración de las cosas sujetas a mudanza.
    \item  \textbf {Sistemas de tiempo predeterminado:}
    \\Conjunto de reglas o metodos para determinar con aticipacion la secuencias de sucesos.
    \item \textbf { Medicion de tiempos de metodos(MTM):}
    \\Es un procedimiento que permite el análisis de todo método manual descomponiéndolo en los movimientos básicos requeridos y asignando a cada movimiento un tiempo estándar predeterminado basado en la naturaleza del movimiento y en las condiciones en las que es realizado.
    \item  \textbf {Muestreo: }
    \\Accion de escoger muestreosque describen de manera exacta las caracteristicas de un conjunto de datos que permitiran deducir y sacar conclusiones del fenomeno a estudiar.
  
    \item  \textbf {Representativo:}
    \\Que sirve para representar algo.
   
    \item \textbf{ Inferir:}
    \\Deducir algo o sacarlo como conclusión de otra cosa. Producir un daño físico o moral.
  
    \item  \textbf {Alcanzar: }
    \\Movimiento realizado con la mano vacia.
    
    \item  \textbf {Mover:}
    \\Movimiento con un objeto en la mano.
   
    \item  \textbf {Muestreo de trabajo: }
    \\Herramienta para disminuir el costo que se presenta en el estudio continuo del tiempo.
    
    \item  \textbf {Estudio de tiempos convencional: }
    \\Es una muestra continua de n ciclos (Suponiendo que la distribución estadística es normal).
    
    \item \textbf{ Estudio de tiempos no convencional:} 
    \\Es una muestra discreta (Suponiendo que la distribución estadística es binomial).
   \item \textbf{MTM: }
    \\Métodos de medición del tiempo.
    \\Medición de tiempos de métodos.
 \item \textbf{Sistemas de tiempo predeterminado (STP): }
    \\ Conjunto de reglas o métodos para determinar con anticipación la secuencia de sucesos.
    
    \item \textbf{Estudio de micromovimientos: }
    \\Division de la asignación de trabajo en therbligs que se logra mediante el analisis, cuadro por cuadro de una pelicula y la mejora de la operación a traves de la eliminación de los movimientos innecesarios y la simplificación de los necesarios.
\end{enumerate}

\bibliographystyle{apalike}
 \bibliography{18/referencias}
 