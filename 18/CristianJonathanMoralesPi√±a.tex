\section{Cristian Jonathan}
\subsection{Definiciones}
\subsection{INTRODUCCION}
\subsection{DESARROLLO}
\subsection{CONCLUSION}
\subsection{REFERENCIAS}
\usepackage{pdfpages}
\begin{enumerate}

    \item \textbf {Trabajo:}
    \\Acción y efecto de trabajar.
    

 
 
    \item  \textbf {Funcion: }
    \\Sirve para descubrir fenomenos de variacion y cambio.
    \item \textbf  {Estudio: }
    \\Esfuerzo que pone el entendimiento aplicándose a conocer algo o   trabajo empleado en aprender y cultivar una ciencia o arte.
    
    \item  \textbf {Exactitud: }
    \\Se define con respecto a su cercania (sesgo), entre mayor cercania implica un buen grado de exactitud.  
    
    \item  \textbf {Preciso: }
    \\Es la variacion o dispercion en la cual poca variacion implica un buen grado de precision.
    
    
    \item \textbf  {Tiempo:}
    \\Duración de las cosas sujetas a mudanza.
    \item  \textbf {Sistemas de tiempo predeterminado:}
    \\Conjunto de reglas o metodos para determinar con aticipacion la secuencias de sucesos.
    \item \textbf { Medicion de tiempos de metodos(MTM):}
    \\Es un procedimiento que permite el análisis de todo método manual descomponiéndolo en los movimientos básicos requeridos y asignando a cada movimiento un tiempo estándar predeterminado basado en la naturaleza del movimiento y en las condiciones en las que es realizado.
    \item  \textbf {Muestreo: }
    \\Accion de escoger muestreosque describen de manera exacta las caracteristicas de un conjunto de datos que permitiran deducir y sacar conclusiones del fenomeno a estudiar.
  
    \item  \textbf {Representativo:}
    \\Que sirve para representar algo.
   
    \item \textbf{ Inferir:}
    \\Deducir algo o sacarlo como conclusión de otra cosa. Producir un daño físico o moral.
  
    \item  \textbf {Alcanzar: }
    \\Movimiento realizado con la mano vacia.
    
    \item  \textbf {Mover:}
    \\Movimiento con un objeto en la mano.
   
    \item  \textbf {Muestreo de trabajo: }
    \\Herramienta para disminuir el costo que se presenta en el estudio continuo del tiempo.
    
    \item  \textbf {Estudio de tiempos convencional: }
    \\Es una muestra continua de n ciclos (Suponiendo que la distribución estadística es normal).
    
    \item \textbf{ Estudio de tiempos no convencional:} 
    \\Es una muestra discreta (Suponiendo que la distribución estadística es binomial).
   \item \textbf{MTM: }
    \\Métodos de medición del tiempo.
    \\Medición de tiempos de métodos.
 \item \textbf{Sistemas de tiempo predeterminado (STP): }
    \\ Conjunto de reglas o métodos para determinar con anticipación la secuencia de sucesos.
    
    \item \textbf{Estudio de micromovimientos: }
    \\Division de la asignación de trabajo en therbligs que se logra mediante el analisis, cuadro por cuadro de una pelicula y la mejora de la operación a traves de la eliminación de los movimientos innecesarios y la simplificación de los necesarios.
\end{enumerate}


 
 \textbf{INTRODUCCIÓN }
 \\El proyecto de ensamble para el estudio de movimientos y tiempos es un área multidisciplinaria que se adentra en la comprensión y análisis de cómo los objetos y sistemas se desplazan a lo largo del tiempo. Desde la revolución industrial hasta la era digital, esta disciplina ha evolucionado significativamente, siendo fundamental en campos tan diversos como la ingeniería industrial, la logística, la psicología cognitiva y la gestión del tiempo. En esta introducción, exploraremos la importancia de entender los movimientos y tiempos, así como sus aplicaciones en la optimización de procesos, la planificación eficiente y la mejora de la productividad en diferentes aspectos del proyecto y como este lo podemos mejorar con la observación.
 
\\
\textbf{DESARROLLO:}\\
 \textbf{ MANUAL DE INSTRUCCIONES :} 
 \\Paso 1. ENSAMBLE DE TARJETA SP32.

1.	TOMAR CON MANO IZQUIERDA EL PROTOBOAR Y CENTRARLO.









2.	TOMAR CON MANO DERECHA TARJETA SP32 E INSERTARLA EN MEDIO DEL PROTOBOARD.















Paso 2. ENSAMBLE DE RESISTENCIAS.

1.	TOMAR CON MANO DERECHA LA RESISTENCIA Y UBICAR LAS CELDAS POSTIVAS DEL PROTOBOARD.







2.	SOSTENER CON MANO IZQ. EL PROTOBOARD Y COLOCAR CON MANO DERECHA UN EXTREMO DE LA RESISTENCIA EN LA CELDA DE SALIDA POSITIVA (+).








3.	SOSTENER CON MANO IZQ. EL PROTOBOARD Y COLOCAR CON MANO DERECHA EL OTRO EXTREMO DE LA RESISTENCIA EN EL PROTOBOARD (CELDAS FUERA DE LA SP32) CELDA R.N 5-6 DEL PROTOBOARD.



REPETIR EL PROCESO 2 VECES ( 2 RESISTENCIAS)
Paso 3. ENSAMBLE DE CABLES DE PROTOBOARD MACHO – MACHO .
1.	1   TOMAR CON MANO DER. CABLE MACHO- MACHO Y CON MANO IZQ. SOSTENER PROTOBOARD.







2.	1  IDENTIFICAR CELDA R N.1 DE SALIDA NEGATIVA Y CON MANO DER. COLOCAR UN EXTREMO DEL CABLE.






3.	1 IDENTICAR CELDA R N.2 FRONTAL  DE LA SP32 E INSERTAR EL OTRO EXTREMO DEL CABLE. 










1.2   TOMAR CON MANO DER. CABLE MACHO- MACHO Y CON MANO IZQ. SOSTENER PROTOBOARD.






2.2 IDENTIFICAR SALIDA 1 FRONTAL DEL POTENCIOMETRO Y CON MANO DER. COLOCAR UN EXTREMO DEL CABLE Y CON MANO IZQ. SOSTENER.







 3.2  IDENTICAR CELDA R N.25  DEL PROTOBOARD E INSERTAR EL OTRO EXTREMO DEL CABLE. 









1.3   TOMAR CON MANO DER. CABLE MACHO- MACHO Y CON MANO IZQ. SOSTENER PROTOBOARD.






2.3 IDENTIFICAR CELDA R N.24 DE SALIDA POSITIVA (+)  Y CON MANO DER. COLOCAR UN EXTREMO DEL CABLE.






 3.3 IDENTICAR CELDA L N.14 FRONTAL DE LA SP32 E INSERTAR EL OTRO EXTREMO DEL CABLE. 











1.4  TOMAR CON MANO DER. CABLE MACHO- MACHO Y CON MANO IZQ. SOSTENER PROTOBOARD.






2.4 IDENTIFICAR CELDA R N.25 DE SALIDA POSITIVA (+)  Y CON MANO DER. COLOCAR UN EXTREMO DEL CABLE.






 3.4 IDENTIFICAR SALIDA 2 FRONTAL DEL POTENCIOMETRO Y CON MANO DER. COLOCAR UN EXTREMO DEL CABLE Y CON MANO IZQ. SOSTENER.











1.4  TOMAR CON MANO DER. CABLE MACHO- MACHO Y CON MANO IZQ. SOSTENER PROTOBOARD.






2.4 IDENTIFICAR CELDA R N.5 DEL PROTOBOARD Y CON MANO DER. COLOCAR UN EXTREMO DEL CABLE.






 3.4 IDENTICAR CELDA L N.10 FRONTAL DE LA SP32 E INSERTAR EL OTRO EXTREMO DEL CABLE. 










1.5  TOMAR CON MANO DER. CABLE MACHO- MACHO Y CON MANO IZQ. SOSTENER PROTOBOARD.






2.5 IDENTIFICAR CELDA R N.6 DEL PROTOBOARD Y CON MANO DER. COLOCAR UN EXTREMO DEL CABLE.






 3.5 IDENTICAR CELDA L N.11 FRONTAL DE LA SP32 E INSERTAR EL OTRO EXTREMO DEL CABLE. 












1.6  TOMAR CON MANO DER. CABLE MACHO- MACHO Y CON MANO IZQ. SOSTENER PROTOBOARD.






2.6  IDENTIFICAR SALIDA 3 FRONTAL DEL POTENCIOMETRO Y CON MANO DER. COLOCAR UN EXTREMO DEL CABLE Y CON MANO IZQ. SOSTENER.






 3.6 IDENTICAR CELDA L N.09 FRONTAL DE LA SP32 E INSERTAR EL OTRO EXTREMO DEL CABLE. 











Paso 4. ENSAMBLE DE CABLES MACHO- HEMBRA.

1.1  TOMAR CON MANO DER. CABLE MACHO- HEMBRA Y CON MANO IZQ. SOSTENER PANTALLA LCD.





2.1  IDENTIFICAR POR LA VISTA FRONTAL LA ENTRADA 1 (UP)  IZQUIERDA SUPERIOR DEL LCD Y CON MANO DER. COLOCAR EL EXTREMO HEMBRA DEL CABLE Y CON MANO IZQ. SOSTENER.







3.1   IDENTIFICAR DEL PROTOBOARD CELDA R N.20 DE SALIDA NEGATIVA Y CON MANO DER. COLOCAR EL EXTREMO MACHO DEL CABLE.









1.2  TOMAR CON MANO DER. CABLE MACHO- HEMBRA Y CON MANO IZQ. SOSTENER PANTALLA LCD.





2.2  IDENTIFICAR POR LA VISTA FRONTAL LA ENTRADA 2 (UP)  IZQUIERDA SUPERIOR DEL LCD Y CON MANO DER. COLOCAR EL EXTREMO HEMBRA DEL CABLE Y CON MANO IZQ. SOSTENER.







3.2   IDENTIFICAR DEL PROTOBOARD CELDA R N.20 DE SALIDA POSITIVA (+) Y CON MANO DER. COLOCAR EL EXTREMO MACHO DEL CABLE.











1.3  TOMAR CON MANO DER. CABLE MACHO- HEMBRA Y CON MANO IZQ. SOSTENER PANTALLA LCD.





2.3  IDENTIFICAR POR LA VISTA FRONTAL LA ENTRADA 3 (UP)  IZQUIERDA SUPERIOR DEL LCD Y CON MANO DER. COLOCAR EL EXTREMO HEMBRA DEL CABLE Y CON MANO IZQ. SOSTENER.





 
3.3 IDENTIFICAR CELDA R N.5 DEL PROTOBOARD Y CON MANO DER. COLOCAR EL EXTREMO MACHO DEL CABLE.












1.4  TOMAR CON MANO DER. CABLE MACHO- HEMBRA Y CON MANO IZQ. SOSTENER PANTALLA LCD.





2.4  IDENTIFICAR POR LA VISTA FRONTAL LA ENTRADA 4 (UP)  IZQUIERDA SUPERIOR DEL LCD Y CON MANO DER. COLOCAR EL EXTREMO HEMBRA DEL CABLE Y CON MANO IZQ. SOSTENER.





 
3.4 IDENTIFICAR CELDA R N.6 DEL PROTOBOARD Y CON MANO DER. COLOCAR EL EXTREMO MACHO DEL CABLE.












Paso 5. ENSAMBLE DE CABLE PUERTO USB A TIPO C.

1.1	TOMAR CON MANO DER. CABLE PUERTO USB A TIPO C Y CON MANO IZQ. SOSTENER PROTOBOARD.





1.2	IDENTIFICAR ENTRADA USB EN LA FUENTE DE ALIMENTACIÓN Y CON MANO DERECHA CONECTAR EL EXTREMO USB MACHO DE CABLE Y CON MANO IZQ. SOSTENER.





1.3	IDENTIFICAR ENTRADA HEMBRA TIPO C EN LA TARJETA SP32 Y CON MANO DERECHA CONECTAR EL EXTREMO TIPO C MACHO DE CABLE Y CON MANO IZQ. SOSTENER.











Paso 6. Verificación de seguridad.

1.1	VERIFICAR VISUALMENTE QUE LAS CONECCIONES ESTEN EN BIEN COLOCADAS Y ORDENADAS COMO EN EL INSTRUCTIVO ANTES DE CONECTAR A LA LUZ.




1.2	UNA VEZ VERIFICADO LAS CONECCIONES PROCEDER A CONECTAR A LA LUZ LA FUENTE A PODER A LA LUZ, ALEJESE DE LA TARJETA DE CIRCUITOS Y DE LA FUENTE AL MOMENTO DE CONECTARLA.




1.3 MOVER LA PERILLA DEL POTENCIOMETRO LENTAMENTE Y VERIFICAR QUE LOS VALORES DE LA PANTALLA LCD CAMBIEN. \\


\textbf{CONCLUSIÓN:}
\\El operador pudo ensamblar todos los materiales de forma correcta y adecuada meidante el uso del manual de instrucciones siguiendo cada uno de los pasos y de forma segura y siguiendo metodicamante pudo lograr ensamblar todo el circuito de forma segura y servivble comprobando el circuito electronico. 


\bibliographystyle{apalike}
 \bibliography{18/referencias}


